\documentclass[final,letterpaper,twoside,12pt]{report}
\usepackage{graphicx}
\renewcommand{\chaptername}{} 
\renewcommand\bibname{References}
\author{Ezequiel Mastrasso}



\begin{document}

\title{\textbf{ezeShader Suite}}
\begin{picture}(0,000)
\put(-120,-660){\includegraphics[scale=.42]{shadersDocumentationImages/ezeCover.jpg}}
\label{pic:ezeShaderSuiteLogo}
\end{picture}


\tableofcontents
%---------------------------------Whats this about--------------------------------------
\chapter{What is this about?}
\title{\textbf{ezeShader Suite}}
\begin{picture}(0,280)
\put(-20,0){\includegraphics[scale=.27]{shadersDocumentationImages/chinchilla_02.jpg}}
\label{pic:ezeCover2}
\end{picture}
\section {goals}
-a beauty render to rul' it all\\
\noindent -enclose the whole lighting pipeline in only a few multipurpose shaders introducing a whole renderman lighting suite ready for your furry needs\\
\noindent -simplify the lighting of characters with fur, including occlussion and gi, to work with 3d environments as well as location footage\\
\noindent -smart raytracing


\begin{picture}(0,300)
\put(-15,0){\includegraphics[scale=.29]{shadersDocumentationImages/ezeWhatsAbout.jpg}}
\label{pic:ezeWhatsAbout}
\end{picture}

\noindent In Figure~{\ref{pic:ezeWhatsAbout}}, from left to right:\\
full beauty render with hair: envMap, occlusion, and global illumination\\
hair diffuse aov pass\\
full beauty pass with hair: envMap, occlusion 2 lights, sss \\
Bottom row: some of the many hair AOVs outputs


\section{key features}
\subsection{ezeHair key features}
-Full raytrace shader\\
-marchsner hair diffuse model\\
\noindent -3delight studio 11 raytrace environmental lights, occlusion and indirect diffuse ready\\
\noindent -2 legacy specular layers (artist friendlier)\\
\noindent -extensive root/tips controls including color, darkening, alpha and ambient and maps for each\\
\noindent -all maps slots support uDim maps type\\

\section{compiling}
\subsection{all platforms}
-shaderdl -I<directory> <shaderName>.sl \\
\\
In directory, supply your \$3DELIGHT/maya/rsl directory, where global\_illumination.h and fluid\_utils.h are found

%--------------------------------------EZEHAIR--------------------------------------------
\chapter{ezeHair}

\begin{picture}(0,290)
\put(-0,0){\includegraphics[scale=.32]{shadersDocumentationImages/ezeHairParams.jpg}}
\label{pic:ezeHairWide}
\end{picture}

\noindent In Figure~{\ref{pic:ezeHairWide}}, ezeHair shader parameters Interface.


\smallskip
\section {linear}
\subsubsection {on/off}
check this on when working in linear space \\
\smallskip
\\
NOTE!: it is recommended to linearize the textures before hand using tdlmake. You using tdl files? Why not linearizing the texture when converting them? Refer to 3delight help on “why not!” using this setting.

%-------------------------------------------------------------
\subsubsection {gamma}
If working In linear space and want to linearize your textures at render time, you can set this to 0.454. But you've been warned not to.

\smallskip

%-------------------------------------------------------------
\section {diffuse}

\begin{picture}(0,290)
\put(0,0){\includegraphics[scale=.42]{shadersDocumentationImages/ezeHairDiffuse.jpg}}
\label{pic:ezeHairDiffuse}
\end{picture}

\noindent In Figure~{\ref{pic:ezeHairDiffuse}}, ezeHair shader AOV\_hair\_diffuse, with a high value wrapIllum Angle (notice the back lighted hair on the back of the neck) and ezeEnvironment. 
Note: Diffuse AOV contains the contribution from all direct lights and ezeEnvironment/ezeIndirect

\subsubsection {Kd}
Kd is the direct lighting multiplier.
\smallskip
\subsubsection {Ki}
the indirect diffuse lighting multiplier (any light contributing to the indirectDiffuse, and environments).
\smallskip


\subsubsection {i\_color}
XXX
\smallskip
\subsubsection {i\_thickness}
hair thickness
\smallskip
\subsubsection {i\_tint}
environment lighting tint
\smallskip
\subsubsection {i\_samples}
trace samples for environment lighting
\smallskip

\subsubsection {i\_xxx\_r (roughness)}
shift of the primary specular toward the the root.
\smallskip
\subsubsection {i\_xxx\_tt (forward scattering)}
a strong forward scattering component from the light coloured hair. This causes blond, brown, gray and white hair to look very bright when lit from behind
\smallskip
\subsubsection {i\_xxx\_r (TRT )}
-coloured secondary peak shifter toward the top from the white primary specular peak.
In a head of hair this leads to the secondary highlight that is visible just above the primary, sometimes appearing more as a coloured fringe on the primary than a separate feature\\
\smallskip


\subsubsection {Tip and root Colors, multipliers, controls and masks}
You can use tip and root colors by themselves. When adding a map, the map will be multiplied by the tip and root colors according with the root and tip position. 
The position will control the ramp start and end points for the texture tip and root blend, saying that the hair strand root is 0 and the tip is 1.\\
If you want to use shaves root/tip colors,leave the maps blank, the colors to white, and check the export root/tip colors in your Shave Globals node.

\begin{picture}(0,210)
\put(0,0){\includegraphics[scale=.53]{shadersDocumentationImages/ezeHairDarkening.jpg}}
\label{pic:ezeHairDarkening.jpg}
\end{picture}

\noindent In Figure~{\ref{pic:ezeHairDarkening.jpg}}, ezeHair shader parameters.


%-------------------------------------------------------------
\section {legacy specular}
\subsubsection {Illum Wrap Angle }
Illum Wrap Angle refers to the light wrapping angle of the light on the hair strand as a cylinder. The bigger the angle the more illumination it will show from side and back lights
\smallskip
Get the maya sample scenes file from the website to get the shift map that you can use for the specular shift.\\
\begin{picture}(0,320)
\put(0,0){\includegraphics[scale=.3]{shadersDocumentationImages/ezeHairSpecularWet.jpg}}
\label{pic:ezeHairSpecularWet}
\end{picture}

\noindent In Figure~{\ref{pic:ezeHairSpecularWet}}, ezeHair shader wet look aov\_hair\_primaryspecular

\subsubsection {overall Specular}
Turns on/off all specular highlights
\smallskip
\subsubsection {primary specular}
multiplies the specular values
\smallskip
\subsubsection {primary specular roughness}
controls the specular roughness
\smallskip
\subsubsection {primary specular Color}
color for the specular (map slot coming soon)
\smallskip
\subsubsection {primary shift}
shifts the specular along the heir lenght
\smallskip
\subsubsection {Using Specular Shifts Maps}
Use the provided map for this parameter.\\
This map is included in the maya scene samples texture/ folder.
\smallskip
%-------------------------------------------------------------
\section {alpha}
If using full tracing, leave the default parameters (override geometry opacity On, and opacity 1 in root and tip for perfomance!!)\\
Either you overriding the Oi, using shaves root/tip transparency, or using a map remember to check Use Surface Shaders in shadow pass!\\
Note:ezeShaders will skip any lighting/sss calculation while not in "finalRenderPass".\\
If you are using any other shaders you might need to use a shaderCollection override for the shadow pass to avoid unnecessary calculations.
\subsubsection {Overriding shave Surface Oi}
You can either use shaves opacity (remember to check tip Fade and export opacity at the shaveShape and shaveGlobals), or override it with the controls under this tab.
\subsubsection {alphaMap}
this map will multiply the opacity values wether you use the Oi or the alpha override parameters.\\
\begin{picture}(0,280)
\put(0,0){\includegraphics[scale=.63]{shadersDocumentationImages/ezeHairAlpha.jpg}}
\label{pic:ezeHairAlpha}
\end{picture}

\noindent In Figure~{\ref{pic:ezeHairAlpha}}, ezeHair shader using alpha overrides and a alpha gradient black \& white map\_hair\_translucence.


\smallskip


%-------------------------------------------------------------
\section {AOVs}
If you plan to use a composition software as Nuke to make your final image, you might want to take advantage of the fact that the multipliers for each effect are not taken into account on the aovs outputs.
For ie: Setting the translucence multiplier to 0.01, it will make it virtually invisible on the final render, but the aov will contain the proper values as calculated in the shader prior to the multiplication.
\subsection {output to standard aovs}
This option will make the shader to output the aovs ALSO to the standard ones (this applies to the ezeSurface
Shader).
\subsection {aov listing}
output varying color aov\_hair\_diffuse\\
output varying color aov\_hair\_indirect\\
output varying color aov\_hair\_surfaceColor\\
output varying color aov\_hair\_totalspecular\\
output varying color aov\_hair\_primaryspecular\\
output varying color aov\_hair\_secondaryspecular\\
output varying color aov\_hair\_subsurface\\
output varying color aov\_hair\_v\\
output varying color aov\_indirect\\
\smallskip

\noindent if output to standard aovs is checked, the shader will also contribute to the common aovs\\
\\
    output varying color aov\_specular = aov\_hair\_totalSpecular\\
    output varying color aov\_diffuse = aov\_hair\_diffuse\\ 
    output varying color aov\_ambient = aov\_hair\_ambient\\
    output varying color aov\_surfaceColor = aov\_hair\_surfaceColor\\
    output varying color aov\_subsurface = aov\_hair\_subsurface\\
    output varying color aov\_indirect = aov\_hair\_indirect\\





\chapter{ezeDisplacement}
\begin{picture}(0,290)
\put(0,0){\includegraphics[scale=.43]{shadersDocumentationImages/ezeDisplace.jpg}}
\label{pic:ezeDisplace}
\end{picture}

\noindent In Figure~{\ref{pic:ezeDisplace}}, ezeDisplacement used to mimic short fur


\section {displace}

Using shave on a displaced geometry? You can easily copy the values from your shave displacement alphaGain and offset map here and it should match!!!

\subsubsection {displacement}
Overall displacement multiplier (usually you would leave this value at 1)
\smallskip
\subsubsection {textureName}
Displacement texture file
\smallskip
\subsubsection {offset}
Here you would usually put your displacement $-\frac{worldSpaceValue}{2}$
\smallskip
\subsubsection {multiplier}
Here you would usually put your displacement $\frac{worldSpaceValue}{2}$
\smallskip
\\
\\
Note: refer to your sculpting software manual on how to get the displacement worldSpaceValue


\chapter{ezehairDisplacement}

To avoid shaves displacement map, and be able to use them also with maya Fur and Yeti this shader will be available soon!

\section {hairDisplace (soon)}

Although not recommend for obvious reasons, this shader might be handy for matching exactly the hair underlying geometry displacement.It has the same attributes as the common displace, but this one will displace the hair geometry on N\_Srf and only update N\_Srf, not N.
Beware of using just the right amount of displacement bound to avoid a render overkill!

\subsubsection {displacement}
Overall displacement multiplier (usually you would leave this value at 1)
\smallskip
\subsubsection {textureName}
Displacement texture file
\smallskip
\subsubsection {offset}
Here you would usually put your displacement $-\frac{worldSpaceValue}{2}$
\smallskip
\subsubsection {multiplier}
Here you would usually put your displacement $\frac{worldSpaceValue}{2}$
\smallskip
\\
\\
Note: refer to your sculpting software manual on how to get the maximum displacement Value in worldSpace 




\chapter{ezeUbber}
\begin{picture}(0,290)
\put(0,0){\includegraphics[scale=.43]{shadersDocumentationImages/ezeUbberWide.jpg}}
\label{pic:ezeUbberWide}
\end{picture}

\noindent In Figure~{\ref{pic:ezeUbberWide}}, ezeHair shader parameters


\begin{picture}(0,290)
\put(0,0){\includegraphics[scale=.2]{shadersDocumentationImages/ezeUbberParameters.jpg}}
\label{pic:ezeUbberParameters}
\end{picture}

\noindent In Figure~{\ref{pic:ezeUbberParameters}}, ezeHair shader parameters


\section {Basic}
\subsubsection {lighttype}
one of "spot", "omni", or "arealight".  Spot lights are  those that point in a particular direction (+z in local light  space, for this light.  Omni lights throw light in all directions.  Area lights are emitted from actual geometry (this only works on BMRT area lights for the time being).\\
\\
NOTE: when using spot type, mayas coneAngle value is discarded and width height parameters in the shader are used instead!\\
\\
NOTE2: if a shadow map is to be used, you need to point to the map (see shadow Mapped subsection), add a attributes node to the light with shadow Maps attributes.\\
See maya tips for tokens or refer to the 3delight docs for a full list of tokens.
\smallskip
\subsubsection {exposure}
overall intensity scaling of the light
\smallskip
\subsubsection {lightcolor}
overall color filtering for the light
\smallskip

%-------------------------------------------------------------
\section {Zshaping and Distance fallOff}
\subsubsection {cuton, cutoff }
define the depth range (z range from the origin, in   light coordinates) over which the light is active.  Outside  this range, no energy is transmitted.
\smallskip
%-------------------------------------------------------------
\subsubsection {nearedge, faredge}
define the width of the transition regions   for the cuton and cutoff.  The transitions will be smooth.
\smallskip
%-------------------------------------------------------------
\subsubsection {falloff }
defines the exponent for falloff.  A falloff of 0 (the  default) indicates that the light is the same brightnessregardless of distance from the source.\\
Falloff==1 indicates  linear (1\/r) falloff\\
falloff==2 indicates 1/r\^2 falloff  (which is physically correct for point-like sources, but  sometimes hard to use)
\smallskip
\subsubsection {falloffdist}
the distance at which the incident energy is actually  equal to intensity*lightcolor.\\
 In other words, the intensity  is actually given by:   I = (falloffdist / distance) \^\\ falloff
\smallskip
\subsubsection {maxintensity}
to prevent the light from becoming unboundedly  large when the distance < falloffdist, the intensity is  smoothly clamped to this maximum value.
\smallskip
\subsubsection {parallelrays }
when 0 (the default), the light appears to eminate  from a single point (i.e. the rays diverge).  When nonzero,  the light rays are parallel, as if from an infinitely distant  source (like the sun).
\smallskip

%-------------------------------------------------------------
\section {Zy Shaping}
Shaping of the cross-section.  The cross-section of the light cone is actually described by a superellipse with the following controls:
\subsubsection {shearx, sheary}
define the amount of shear applied to the light  cone direction.  Default is 0, meaning that the center of the  light cone is aligned with the z axis in local light space
\smallskip
\subsubsection {width, height}
define the amount of shear applied to the light  cone direction.  Default is 0, meaning that the center of the  light cone is aligned with the z axis in local light space
\smallskip
\subsubsection {wedge, hedge}
the amount of width and height edge fuzz,  respectively.  Values of 0 will make a sharp cutoff, larger  values (up to 1) will make the edge softer.
\smallskip
\subsubsection {roundness}
controls how rounded the corners of the superellipse  are.  If this value is 0, the cross-section will be a perfect  rectangle.  If the value is 1, the cross-section will be a  perfect circle.  In between values control the roundness of  the corners in a fairly obvious way.
\smallskip
\subsubsection {beamdistribution }
controls intensity falloff due to angle.  A value of 0 (the default) means no angle falloff.  A value of 1 is roughly physically correct for a spotlight, and  corresponds to a cosine falloff.  For a BMRT area light, the cosine falloff happens automatically, so 0 is the right physical  value to use.  In either case, you may use larger values to  make the spot more bright in the center than the outskirts.  This parameter has no effect for omni lights.
\\
Note:When using this parameter with a hair backLight,no matter the wrapIllum Angle of the ezeHair diffuse, It will draw the light useless! 
\smallskip

%-------------------------------------------------------------
\section {gobo}
Cookie or slide filter
\subsubsection {slidename}
if a filename is supplied, a texture lookup will be  done and the light emitted from the source will be filtered  by that color, much like a slide projector.  If you want to  make a texture map that simply blocks light, just make it  black-and-white, but store it as a RGB texture.  For  simplicity, the shader assumes that the texture file will  have at least three channels.
\smallskip
%-------------------------------------------------------------
\section {Noise}
\subsubsection {Projected noise on the light}
\smallskip
%-------------------------------------------------------------
\subsubsection {noiseamp}
amplitude of the noise.  A value of 0 (the default)  means not to use noise.  Larger values increase the blotchiness  of the projected noise
\smallskip
%-------------------------------------------------------------
\subsubsection {noisefreq }
frequency of the noise.
\smallskip
%-------------------------------------------------------------
\subsubsection {noiseoffset }
spatial offset of the noise.  This can be animated,  for example if you are using the noise to simulate the attenuation of light as it passes through a window with  water drops dripping down it.
\smallskip
%-------------------------------------------------------------
\section {Shadow Mapped}
shadows are mainly computed by shadow maps.  Please consult the PRMan documentation for more information on the meanings of these parameters
\subsubsection {shadowmap }
the name of the texture containing the shadow map.  If  this value is "" (the default), no shadow map will be used. 
\smallskip
%-------------------------------------------------------------
\subsubsection {shadowblur  }
how soft to make the shadow edge, expressed as a  percentage of the width of the entire shadow map.
\smallskip
%-------------------------------------------------------------
\subsubsection {shadowbias  }
the amount of shadow bias to add to the lookup.
\smallskip
%-------------------------------------------------------------
\subsubsection {shadownsamps  }
the number of samples to use.
\smallskip
%-------------------------------------------------------------
\section {RayTraced}
These options work only for BMRT:   raytraceshadow - if nonzero, cast a ray to see if we are in shadow.  The default is zero, i.e. not to try raytracing.
\subsubsection {Ray-traced shadows}
check this on when working in linear space 
\smallskip
%-------------------------------------------------------------
\subsubsection {shadowcheat}
add this offset to the light source position.  This  allows you to cause the shadows to eminate as if the light  were someplace else, but without changing the area  illuminated or the appearance of highlights, etc.
\smallskip
%-------------------------------------------------------------
\section {fake Blocker Shadow}
"Fake" shadows from a blocker object.  A blocker is a superellipse in 3-space which effectively blocks light.  But it's not really geometry, the shader just does the intersection with the superellipse.  The blocker is defined to lie on the x-y plane of its own coordinate system (which obviously needs to be defined in the RIB file using the CoordinateSystem command).
\subsubsection {blockercoords}
the name of the coordinate system that defines the  local coordinates of the blocker.  If this is "", it indicates  that the shader should not use a blocker at all.
\smallskip
\subsubsection {blockerwidth, blockerheight }
define the dimensions of the blocker's  superellipse shape.
\smallskip
\subsubsection {blockerwedge, blockerhedge }
define the fuzzyness of the edges.
\smallskip
\subsubsection {blockerround}
how round the corners of the blocker are (same  control as the "roundness" parameter that affects the light  cone shape.
\smallskip
%-------------------------------------------------------------
\subsubsection {shadowcolor }
Shadows (i.e. those regions with "occlusion" as  defined by any or all of the shadow map, ray cast, or  blocker) don't actually have to block light.  In fact, in  this shader, shadowed regions actually just change the color  of the light to "shadowcolor".  If this color is set to  (0,0,0), it effectively blocks all light.\\
But if you set it  to, say (.25,.25,.25), it will make the shadowed regions lose  their full brightness, but not go completely dark.  Another  use is if you are simulating sunlight: set the lightcolor to  something yellowish, and make the shadowcolor dark but  somewhat bluish.  Another effect of shadows is to set the  \_\_nonspecular flag, so that the shadowed regions are lit only  diffusely, without highlights.
\smallskip
%-------------------------------------------------------------
\section {Contribution controls}
Allows you to control the effect amount a light will have on the shader.\\
0 means full amount, 0.2 = 80\%, 0.9 = 10\%, 1 = None, 
\subsubsection {\_\_nonspecular }
when set to 1, this light does not create  specular highlights!  The default is 0, which means it makes highlights just fine (except for regions in shadows, as  explained above).  This is very handy for lights that are meant to be fill lights, rather than key lights.  NOTE: This depends on the surface shader looking for, and correctly acting upon, this parameter.  The builtin functions  diffuse(), specular() and phong() all do this, for PRMan 3.5  and later, as well as BMRT 2.3.5 and later.  But if you write  your own illuminance loops in your surface shader, you've got  to account for it yourself The PRMan user manual explains how  to do this.
\smallskip
\subsubsection {\_\_nondiffuse }
the analog to \_\_nonspecular, if this flag is set to  1, this light will only cast specular highlights, but not  diffuse light.  This is useful for making a light that only  makes specular highlights, without affecting the rest of the  illumination in the scene.  All the same caveats apply with  respect to the surface shader, as described above for \_\_nonspecular
\smallskip

\subsubsection {\_\_nonScatter}
scatter Effects, works for both ezeSurface and ezeHair
\smallskip

\subsubsection {\_\_nonTransluncence}
only used in ezeHair
\smallskip

\subsubsection {\_\_nonSubsurface}
subSurface effect, works for both ezeSurface and ezeHair
\smallskip

\subsubsection {\_\_nonSpecularMask}
blends the specular masks, any value between 0-1 will have an effect on the mask transparency.
See ezeSurface Specular masks section for more information on this
\smallskip

\subsubsection {\_\_foglight  }
the "noisysmoke" shader distributed with BMRT will add  atmospheric scattering only for those lights that have this  parameter set to 1 (the default).  In other words, if you use  this light with noisysmoke, you can set this flag to 0 to  make a particular light *not* cause illumination in the fog.  Note that the noisysmoke shader is distributed with BMRT, but  will also work just fine with PRMan (3.7 or later).
\smallskip





%--------------------------------------uDim--------------------------------------------
\chapter{uDim reference}
All texture map paramters can use a common texture map, or uv spanned textures like the ones mari does when working with different uvspaces.
\smallskip
\section {parameters}
\subsubsection {on/off}
This will override the usage of the texture map and use the uDim instead
\smallskip
%-------------------------------------------------------------
\subsubsection {texName}
This is the actual path+filename without uv space coord, frameNumber, nor extension.
\smallskip
\subsubsection {diffuse Maximun U}
This is the number of uv spaces to use accross u, if reached u will be 0 again and v=(v+1), see example images.
\smallskip
\subsubsection {diffuse frame number}

\smallskip
\subsubsection {useVariant}
Your file name may be something like this fileName\_variant . uvSpace N. ext, where variant is a version, a type, or a resolution count of the file.
\smallskip
\subsubsection {variant name}
Here you can specify the variant name to use if needed. for Ie "2k" (diffuse\_2k.0001.tdl), "8k" (diffuse\_8k.0001.tdl), or even a date like "20120107" (specular\_20120107.0001.tdl)
\smallskip
\subsubsection {useAnimMap}
Inserts the frame number with a 4 digit padding between the complete file name and the extension like diffuse\_8k.0054.0001.tdl
\subsubsection {diffuseFileExt}
file extension to be used, either tdl of tiff.
Remember tiff files can have a dramatic impact on your render times specially with big files! i mean seriously dramatic.
\smallskip

\chapter{Maya quick start setup}
Some Object sets are suitable for speeding up rendering of the passes. Not required though.

The passes you will need to quickly get started are:

Shadow pass

\section {Passes}
%-------------------------------------------------------------
\subsubsection {Shadow pass}
\begin{tabular}{ll}
one line& \parbox[t]{5cm}{another\\column}\\
second line here& and here
\end{tabular}

\footnotesize
\begin{tabular}{|l|l|l|}
\hline 
Object Sets & pass\_ShadowGeo & all geometry you want it to cast shadows  \\ 
\hline 
• & pass\_shadowLights & all lights you want to calculate shadows from \\ 
\hline 
render Settings & • & 
\parbox[t]{5cm}{Basic pass, + this attributes:\\
-display sets: geo and lights\\
-render shadows: On\\
-use displacement shaders: On\\
-full volumetric shaders: On\\
-export all AOVS: On\\}
\\ 
\hline 
\end{tabular} 

\smallskip
%-------------------------------------------------------------
\subsubsection {Bake Pass}
\footnotesize
\begin{tabular}{|l|l|l|}
\hline 
Object Sets & pass\_bakeGeo & \parbox[t]{5cm}{all geometry you want it to\\
occlude ambient lighting\\}\\
\hline
Shader collection overrides & \parbox[t]{5cm}{bakeGeoAttributes\\
ezeBake shader}
 & \parbox[t]{5cm}{-two sided\\
-raster oriented dice Off}\\
\hline 
render Settings & • & 
\parbox[t]{5cm}{Basic pass, + this attributes:\\
-display sets: geo and lights\\
-render shadows: Off\\
-use displacement shaders: On\\
-full volumetric shaders: On\\
-export all AOVS: On\\}
\\ 
\hline 
\end{tabular} 
\smallskip




%-------------------------------------------------------------
\subsubsection {indirectBakePass}
\footnotesize
\begin{tabular}{|l|l|l|}
\hline 
Object Sets & pass\_bakeIndirectGeo & \parbox[t]{5cm}{all geometry you want to have
indirect illuminationthis should NOT include the hair.  The hair will sample this 3d
texture bake from the strand root
 position in 3d pace to get the correct values}\\
\hline
Shader collection overrides & \parbox[t]{5cm}{bakeGeoAttributes\\
ezeBake shader}
 & \parbox[t]{5cm}{-two sided\\
-raster oriented dice Off}\\
\hline 
render Settings & • & 
\parbox[t]{5cm}{Basic pass, + this attributes:\\
-display sets: geo and lights\\
-render shadows: Off\\
-use displacement shaders: On\\
-full volumetric shaders: On\\
-export all AOVS: On\\}
\\ 
\hline 
\end{tabular} 
\smallskip


%-------------------------------------------------------------
\subsubsection {fullRendePass}
\footnotesize
\begin{tabular}{|l|l|l|}
\hline 
Object Sets & pass\_fullRenderGeo  & all geometry you want to render\\
\hline 
• & pass\_fullRenderLights  & all lights you want to be included while render\\
\hline
Geometry Attributes & • & \parbox[t]{7cm}{hairGeoAttributes\\
-dice type Hair\\
-primitive Hair\\
-shading rate: 40 (this will speed up renders\\
when sampling, as micropolygon dicing is not\\
needed for hair primitives\\
-raster oriented dice Off\\
geometryAttributes:\\
---}
\\
\hline 
render Settings & • & 
\parbox[t]{5cm}{Full pass with this settings:\\
renderShadows: Off}
\\ 
\hline 
\end{tabular} 


\chapter{Maya tips}

\section {ezeUber maya visualization plugin}
Check this python plugin to interactively see the uberLight parameters in the viewport\\
http://www.creativecrash.com/maya/downloads/scripts-plugins/rendering/renderman/c/uberlighthelper

\section {file path expressions}
%-------------------------------------------------------------
\subsubsection {most used tokens}
Refer to File Path Expressions section in 3delight for maya help for complete token listing\\

\begin{tabular}{ll}
\$VAR or \%VAR\% & The value of the environment variable ‘VAR’ will replace \$VAR in the path.\\
@ & The current frame number will replace the @ character.\\
\# & The current frame number, padded to form a 4 digits number, will replace the\\
• & \#
character.\\
`melCommand`  & This string will be replaced with the path to the current Maya project directory.\\
\textless project\textgreater & This string will be replaced with the path to the current Maya project directory.\\
\textless pass\textgreater & This string will be replaced with the name of the render pass node used for rendering\\
\textless scene\textgreater & This string will be replaced with the name of the scene\\
\textless camera\textgreater & This string will be replaced with the name of the camera shape being used\\
• & for rendering\\
\textless aov\textgreater & This string will be replaced with the name of the variable being output in the display.\\
• &Only valid in render pass display filename attributes\\
\end{tabular}



\begin{thebibliography}{99}
%the '99' allows you up to 99 different references. 
\bibitem{c:blackBody} blackbody. Based on Mitchell Charitys rgb blackbody color mappings\\
http://www.vendian.org/mncharity/dir3/blackbody/ \\
http://www.vendian.org/mncharity/dir3/blackbody/intensity.html \\
http://www.vendian.org/mncharity/dir3/blackbody/UnstableURLs/bbr\_color.html \\
\bibitem{c:uDim} uDim. Code based on William J. Earl\\
http://earlyworm.org/2011/udim-constant-shader/
\bibitem{c:hairSpecularScatter} hair specular and scatter Code based Sachin Shrestha from 3delight forums\\
\bibitem{c:specularMask} surface specular masks Code rewritten but based on Charles Trippe ct\_surf\_ceramic shader\\
\bibitem{c:uberLight} ezeUber light based on Larry Gritz uberLight shader \\
\\
\noindent other Resources
\bibitem{c:otherResources}
http://www.renderman.org/RMR/Publications/sig06.course25.pdf\\
\end{thebibliography}



\end{document}
